\usepackage[T1]{fontenc}
\usepackage[utf8x]{inputenc}
\usepackage[activeacute,spanish,es-tabla]{babel} % Idioma español, con sus configuraciones
\usepackage[left=2cm, right=2cm, bottom=3cm, top=3cm, headheight=40pt]{geometry} % Márgenes
\usepackage{graphicx} % Required for including pictures
\usepackage{float} % Allows putting an [H] in \begin{figure} to specify the exact location of the figure
\usepackage{wrapfig} % Allows in-line images
\usepackage[nottoc, notlot, notlof]{tocbibind} % Índices, sin ToC, LoF ni LoT
\renewcommand{\refname}{Bibliografía} % Nombre para Bibliografía (clase article)
%\renewcommand{\bibname}{Bibliografía} % Nombre para Bibliografía (clase book/report)
\renewcommand\tocbibname{Bibliografía}
\usepackage{amsmath}
\usepackage{amsfonts}
\usepackage{amssymb}
\usepackage{siunitx} % Unidades del SI
\sisetup{output-decimal-marker = {,}}
\usepackage{cancel} % Permite cancelar (tachar) elementos
\usepackage{tabu} % Tablas chéveres
\usepackage{booktabs} % Allows the use of \toprule, \midrule and \bottomrule in tables for horizontal lines
\usepackage{multirow} % Celdas en más de una fila
\usepackage{easybmat} % Matrices por Bloques
\usepackage{lipsum} % Lorem Ipsum

\title{Template Informes 2.0} % Nombre archivo en Overleaf
\newcommand{\titulo}{Título del informe}
\newcommand{\ramo}{XXNNNN Ramo}
\newcommand{\departamento}{Departamento de Ingeniería XXXXX}

\usepackage{color}
\definecolor{gray51}{rgb}{0.51,0.51,0.51}
\definecolor{gray71}{rgb}{0.71,0.71,0.71}
\newcommand{\HRule}{\rule{\linewidth}{.4mm}}

\usepackage{listings} % Incluye códigos
